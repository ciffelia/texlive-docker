\chapter{緒言}

本研究の研究対象は重要である.いまこんな感じの情勢なのでこんなことが求められている.既存手法にはこんなものがあるが,こんな課題がある.先行研究にこんな課題解決のヒントがあって,いま注目されている.本研究は,そんな先行研究を受けてこんな感じの新しいアプローチを試す.あんなことやこんなことをやった.するとこんな感じの結果が得られるから,本研究スゴいよ.

本論分は以上の内容を取りまとめたものであり,以下に示す5章より構成される.

第1章は緒言であり,本研究の背景,目的及び概要について述べたものである.

第2章ではもうちょっと詳しい背景とか先行研究紹介しつつ,その上に本研究がどう成り立っているか提案手法と原理を説明する.

第3章ではトピックを分けたい原理をここに追加したり,提案手法の理論解析を行ったりする.そうじゃないならもう実験入る.

第4章ではこんな実験してこんな結果が得られる.こんなことを示す.提案手法の結果を評価する.

第5章は結言である.

以上,本論文の企図するところを概説した.
