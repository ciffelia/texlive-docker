%%%%%%%%%%%%%%%%%%%%%%%%%%%%%%%%%%%%%%%%%%%%%%
\chapter{実験}
\label{sec:gradient_methods}
%%%%%%%%%%%%%%%%%%%%%%%%%%%%%%%%%%%%%%%%%%%%%%

\section{まえがき}
本章では,こんな感じの実験を行う.

\section{実験に使用したアルゴリズム}
\label{sec:algorithm}
本実験で用いるアルゴリズムを Algorithm \ref{alg1} に示す.Algorithm \ref{alg1} はここまでで説明した事項を受けてこんな感じに実装した.

\section{実験に用いるデータセット}
実験で用いるデータはこうやって用意した.作った.ダウンロードした.

\section{評価に用いる指標}
\label{sec:criteria}
本実験ではこんな項目を評価した.

\section{実験結果と評価}
\label{sec:evaluation}
本実験ではこんな結果になったのでこんなことが言える.

\begin{figure}[!t]
  \begin{algorithm}[H]
    \caption{FizzBuzz}
    \label{alg1}
    \begin{algorithmic}[1]
      \Require max number N \par
        \hskip \algorithmicindent {\hspace*{-0.7em}} some other input A \par
        \hskip \algorithmicindent {\hspace*{-0.7em}} some other input B
      \Ensure output Y \par
        \hskip \algorithmicindent {\hspace*{-0.7em}} some other output Z
        \For{$k=1,2,\cdots,N$}
          \If {$i \% 15 == 0$}
            \State print('FizzBuzz')
	  \ElsIf {$i \% 3 == 0$}
            \State print('Fizz')
	  \ElsIf {$i \% 5 == 0$}
            \State print('Buzz')
	  \Else
            \State print(i)
	  \EndIf
        \EndFor
    \end{algorithmic}
  \end{algorithm}
\end{figure}


